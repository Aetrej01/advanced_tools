% Options for packages loaded elsewhere
\PassOptionsToPackage{unicode}{hyperref}
\PassOptionsToPackage{hyphens}{url}
%
\documentclass[
]{article}
\usepackage{amsmath,amssymb}
\usepackage{iftex}
\ifPDFTeX
  \usepackage[T1]{fontenc}
  \usepackage[utf8]{inputenc}
  \usepackage{textcomp} % provide euro and other symbols
\else % if luatex or xetex
  \usepackage{unicode-math} % this also loads fontspec
  \defaultfontfeatures{Scale=MatchLowercase}
  \defaultfontfeatures[\rmfamily]{Ligatures=TeX,Scale=1}
\fi
\usepackage{lmodern}
\ifPDFTeX\else
  % xetex/luatex font selection
\fi
% Use upquote if available, for straight quotes in verbatim environments
\IfFileExists{upquote.sty}{\usepackage{upquote}}{}
\IfFileExists{microtype.sty}{% use microtype if available
  \usepackage[]{microtype}
  \UseMicrotypeSet[protrusion]{basicmath} % disable protrusion for tt fonts
}{}
\makeatletter
\@ifundefined{KOMAClassName}{% if non-KOMA class
  \IfFileExists{parskip.sty}{%
    \usepackage{parskip}
  }{% else
    \setlength{\parindent}{0pt}
    \setlength{\parskip}{6pt plus 2pt minus 1pt}}
}{% if KOMA class
  \KOMAoptions{parskip=half}}
\makeatother
\usepackage{xcolor}
\usepackage[margin=1in]{geometry}
\usepackage{color}
\usepackage{fancyvrb}
\newcommand{\VerbBar}{|}
\newcommand{\VERB}{\Verb[commandchars=\\\{\}]}
\DefineVerbatimEnvironment{Highlighting}{Verbatim}{commandchars=\\\{\}}
% Add ',fontsize=\small' for more characters per line
\usepackage{framed}
\definecolor{shadecolor}{RGB}{248,248,248}
\newenvironment{Shaded}{\begin{snugshade}}{\end{snugshade}}
\newcommand{\AlertTok}[1]{\textcolor[rgb]{0.94,0.16,0.16}{#1}}
\newcommand{\AnnotationTok}[1]{\textcolor[rgb]{0.56,0.35,0.01}{\textbf{\textit{#1}}}}
\newcommand{\AttributeTok}[1]{\textcolor[rgb]{0.13,0.29,0.53}{#1}}
\newcommand{\BaseNTok}[1]{\textcolor[rgb]{0.00,0.00,0.81}{#1}}
\newcommand{\BuiltInTok}[1]{#1}
\newcommand{\CharTok}[1]{\textcolor[rgb]{0.31,0.60,0.02}{#1}}
\newcommand{\CommentTok}[1]{\textcolor[rgb]{0.56,0.35,0.01}{\textit{#1}}}
\newcommand{\CommentVarTok}[1]{\textcolor[rgb]{0.56,0.35,0.01}{\textbf{\textit{#1}}}}
\newcommand{\ConstantTok}[1]{\textcolor[rgb]{0.56,0.35,0.01}{#1}}
\newcommand{\ControlFlowTok}[1]{\textcolor[rgb]{0.13,0.29,0.53}{\textbf{#1}}}
\newcommand{\DataTypeTok}[1]{\textcolor[rgb]{0.13,0.29,0.53}{#1}}
\newcommand{\DecValTok}[1]{\textcolor[rgb]{0.00,0.00,0.81}{#1}}
\newcommand{\DocumentationTok}[1]{\textcolor[rgb]{0.56,0.35,0.01}{\textbf{\textit{#1}}}}
\newcommand{\ErrorTok}[1]{\textcolor[rgb]{0.64,0.00,0.00}{\textbf{#1}}}
\newcommand{\ExtensionTok}[1]{#1}
\newcommand{\FloatTok}[1]{\textcolor[rgb]{0.00,0.00,0.81}{#1}}
\newcommand{\FunctionTok}[1]{\textcolor[rgb]{0.13,0.29,0.53}{\textbf{#1}}}
\newcommand{\ImportTok}[1]{#1}
\newcommand{\InformationTok}[1]{\textcolor[rgb]{0.56,0.35,0.01}{\textbf{\textit{#1}}}}
\newcommand{\KeywordTok}[1]{\textcolor[rgb]{0.13,0.29,0.53}{\textbf{#1}}}
\newcommand{\NormalTok}[1]{#1}
\newcommand{\OperatorTok}[1]{\textcolor[rgb]{0.81,0.36,0.00}{\textbf{#1}}}
\newcommand{\OtherTok}[1]{\textcolor[rgb]{0.56,0.35,0.01}{#1}}
\newcommand{\PreprocessorTok}[1]{\textcolor[rgb]{0.56,0.35,0.01}{\textit{#1}}}
\newcommand{\RegionMarkerTok}[1]{#1}
\newcommand{\SpecialCharTok}[1]{\textcolor[rgb]{0.81,0.36,0.00}{\textbf{#1}}}
\newcommand{\SpecialStringTok}[1]{\textcolor[rgb]{0.31,0.60,0.02}{#1}}
\newcommand{\StringTok}[1]{\textcolor[rgb]{0.31,0.60,0.02}{#1}}
\newcommand{\VariableTok}[1]{\textcolor[rgb]{0.00,0.00,0.00}{#1}}
\newcommand{\VerbatimStringTok}[1]{\textcolor[rgb]{0.31,0.60,0.02}{#1}}
\newcommand{\WarningTok}[1]{\textcolor[rgb]{0.56,0.35,0.01}{\textbf{\textit{#1}}}}
\usepackage{longtable,booktabs,array}
\usepackage{calc} % for calculating minipage widths
% Correct order of tables after \paragraph or \subparagraph
\usepackage{etoolbox}
\makeatletter
\patchcmd\longtable{\par}{\if@noskipsec\mbox{}\fi\par}{}{}
\makeatother
% Allow footnotes in longtable head/foot
\IfFileExists{footnotehyper.sty}{\usepackage{footnotehyper}}{\usepackage{footnote}}
\makesavenoteenv{longtable}
\usepackage{graphicx}
\makeatletter
\def\maxwidth{\ifdim\Gin@nat@width>\linewidth\linewidth\else\Gin@nat@width\fi}
\def\maxheight{\ifdim\Gin@nat@height>\textheight\textheight\else\Gin@nat@height\fi}
\makeatother
% Scale images if necessary, so that they will not overflow the page
% margins by default, and it is still possible to overwrite the defaults
% using explicit options in \includegraphics[width, height, ...]{}
\setkeys{Gin}{width=\maxwidth,height=\maxheight,keepaspectratio}
% Set default figure placement to htbp
\makeatletter
\def\fps@figure{htbp}
\makeatother
\setlength{\emergencystretch}{3em} % prevent overfull lines
\providecommand{\tightlist}{%
  \setlength{\itemsep}{0pt}\setlength{\parskip}{0pt}}
\setcounter{secnumdepth}{-\maxdimen} % remove section numbering
\usepackage{booktabs}
\usepackage{longtable}
\usepackage{array}
\usepackage{multirow}
\usepackage{wrapfig}
\usepackage{float}
\usepackage{colortbl}
\usepackage{pdflscape}
\usepackage{tabu}
\usepackage{threeparttable}
\usepackage{threeparttablex}
\usepackage[normalem]{ulem}
\usepackage{makecell}
\usepackage{xcolor}
\usepackage{siunitx}

  \newcolumntype{d}{S[
    input-open-uncertainty=,
    input-close-uncertainty=,
    parse-numbers = false,
    table-align-text-pre=false,
    table-align-text-post=false
  ]}
  
\ifLuaTeX
  \usepackage{selnolig}  % disable illegal ligatures
\fi
\IfFileExists{bookmark.sty}{\usepackage{bookmark}}{\usepackage{hyperref}}
\IfFileExists{xurl.sty}{\usepackage{xurl}}{} % add URL line breaks if available
\urlstyle{same}
\hypersetup{
  pdftitle={ICA 5},
  pdfauthor={Alice Trejo},
  hidelinks,
  pdfcreator={LaTeX via pandoc}}

\title{ICA 5}
\author{Alice Trejo}
\date{2024-03-10}

\begin{document}
\maketitle

\hypertarget{customer-churn}{%
\subsection{Customer Churn}\label{customer-churn}}

In this assignment, you will estimate a survival analysis model on
customer churn. In this dataset, you will find various characteristics
about a set of phone company customers

customerID: Customer ID number gender: Provides the stated gender of the
customer SeniorCitizen: States if the person is a senior citizen
Partner: Does the person have a partner Dependents: Does the person have
dependents tenure: States how long the person has been with the Bank
PhoneService: Do they have phone service MultipleLines: Do they have
multiple lines InternetService: What type of Internet Service do they
have (DSL, fiber optic, none) OnlineSecurity: Do they have online
security? OnlineBackup: Do they have online backup? DeviceProtection: Do
they have device protection? TechSupport: Did they use tech support?
StreamingTV: Do they use the internet to stream TV? StreamingMovies: Do
they use the internet to stream movies? Contract: What type of contract
do they have (month-to-month, one-year, or two year) PaperlessBilling:
Do they use paperless biling PaymentMethod: How do they pay for services
MonthlyCharges: What is their monthly charge? TotalCharges: What is
their total charge for the quarter? Churn: Did they leave the company

\begin{enumerate}
\def\labelenumi{\arabic{enumi})}
\tightlist
\item
  Find the simple average of \texttt{tenure}. Explain why this simple
  average can be biased.
\end{enumerate}

\begin{Shaded}
\begin{Highlighting}[]
\FunctionTok{library}\NormalTok{(readr)}
\NormalTok{Churn }\OtherTok{\textless{}{-}} \FunctionTok{read\_csv}\NormalTok{(}\StringTok{"WA\_Fn{-}UseC\_{-}Telco{-}Customer{-}Churn.csv"}\NormalTok{)}
\end{Highlighting}
\end{Shaded}

The average time a customer spends with the bank is 32.37. ANS: The
simple average of customer tenure can be misleading due to several
biases. Survivorship bias can overestimate average tenure as long-term
customers are more likely to appear in the data. Right censoring
introduces uncertainty as we don't know the total tenure of customers
who haven't churned by the time the data was collected. The average
tenure can be skewed downwards if the company has been growing and
acquiring more customers over time, leading to a larger number of
customers with shorter tenures. Cohort effects can also influence
average tenure, as customers who joined at different times may have
different average tenures. These factors suggest that it's often more
informative to look at the distribution of tenure or to model churn
directly using survival analysis techniques.

\begin{enumerate}
\def\labelenumi{\arabic{enumi})}
\setcounter{enumi}{1}
\tightlist
\item
  Find the simple average of \texttt{tenure} by the following groups.
  Gender, SeniorCitizen, and Partner.
\end{enumerate}

\begin{Shaded}
\begin{Highlighting}[]
\FunctionTok{library}\NormalTok{(knitr)}
\CommentTok{\#aggregate(Churn$tenure,by=list(Churn$gender,churn$SeniorCitizen,Churn$Partner),FUN=mean)}
\NormalTok{tab1}\OtherTok{\textless{}{-}}\FunctionTok{aggregate}\NormalTok{(Churn}\SpecialCharTok{$}\NormalTok{tenure,}\AttributeTok{by=}\FunctionTok{list}\NormalTok{(Churn}\SpecialCharTok{$}\NormalTok{gender,Churn}\SpecialCharTok{$}\NormalTok{SeniorCitizen,Churn}\SpecialCharTok{$}\NormalTok{Partner),}\AttributeTok{FUN=}\NormalTok{mean)}
\FunctionTok{names}\NormalTok{(tab1)}\OtherTok{\textless{}{-}}\FunctionTok{c}\NormalTok{(}\StringTok{"Gender"}\NormalTok{, }\StringTok{"Senior Citizen"}\NormalTok{, }\StringTok{"Partnered"}\NormalTok{, }\StringTok{"Mean Tenure"}\NormalTok{)}
\FunctionTok{kable}\NormalTok{(tab1)}
\end{Highlighting}
\end{Shaded}

\begin{longtable}[]{@{}lrlr@{}}
\toprule\noalign{}
Gender & Senior Citizen & Partnered & Mean Tenure \\
\midrule\noalign{}
\endhead
\bottomrule\noalign{}
\endlastfoot
Female & 0 & No & 22.87432 \\
Male & 0 & No & 23.18970 \\
Female & 1 & No & 24.85938 \\
Male & 1 & No & 25.37751 \\
Female & 0 & Yes & 41.72639 \\
Male & 0 & Yes & 42.55436 \\
Female & 1 & Yes & 42.63710 \\
Male & 1 & Yes & 40.54154 \\
\end{longtable}

\begin{Shaded}
\begin{Highlighting}[]
\CommentTok{\#library(modelsummary)}
\CommentTok{\#datasummary(tenure\textasciitilde{}Mean*(gender), data=Churn)}
\CommentTok{\#datasummary(tenure\textasciitilde{}Mean*(SeniorCitizen), data=Churn)}
\CommentTok{\#datasummary(tenure\textasciitilde{}Mean*(gender)+Mean*(Partner), data = Churn)}
\end{Highlighting}
\end{Shaded}

\begin{enumerate}
\def\labelenumi{\arabic{enumi})}
\setcounter{enumi}{2}
\tightlist
\item
  Find the simple average of \texttt{tenure} and \texttt{MonthlyCharge}
  by \texttt{Contract} type.
\end{enumerate}

\begin{Shaded}
\begin{Highlighting}[]
\FunctionTok{library}\NormalTok{(modelsummary)}
\end{Highlighting}
\end{Shaded}

\begin{verbatim}
## Version 2.0.0 of `modelsummary`, to be released soon, will introduce a
##   breaking change: The default table-drawing package will be `tinytable`
##   instead of `kableExtra`. All currently supported table-drawing packages
##   will continue to be supported for the foreseeable future, including
##   `kableExtra`, `gt`, `huxtable`, `flextable, and `DT`.
##   
##   You can always call the `config_modelsummary()` function to change the
##   default table-drawing package in persistent fashion. To try `tinytable`
##   now:
##   
##   config_modelsummary(factory_default = 'tinytable')
##   
##   To set the default back to `kableExtra`:
##   
##   config_modelsummary(factory_default = 'kableExtra')
\end{verbatim}

\begin{Shaded}
\begin{Highlighting}[]
\FunctionTok{datasummary}\NormalTok{((}\StringTok{\textasciigrave{}}\AttributeTok{Tenure}\StringTok{\textasciigrave{}}\OtherTok{=}\NormalTok{tenure)}\SpecialCharTok{+}\NormalTok{(}\StringTok{\textasciigrave{}}\AttributeTok{Monthly Charges}\StringTok{\textasciigrave{}}\OtherTok{=}\NormalTok{MonthlyCharges)}\SpecialCharTok{\textasciitilde{}}\NormalTok{Mean}\SpecialCharTok{*}\NormalTok{(Contract), }\AttributeTok{data =}\NormalTok{ Churn)}
\end{Highlighting}
\end{Shaded}

\begin{table}
\centering
\begin{tabular}[t]{lrrr}
\toprule
  & Month-to-month & One year & Two year\\
\midrule
Tenure & \num{18.04} & \num{42.04} & \num{56.74}\\
Monthly Charges & \num{66.40} & \num{65.05} & \num{60.77}\\
\bottomrule
\end{tabular}
\end{table}

\begin{enumerate}
\def\labelenumi{\arabic{enumi})}
\setcounter{enumi}{3}
\tightlist
\item
  Estimate a Kaplan Meier survival model. Use gender and senior citizen
  as explanatory variables. Do these variables produce statistically
  different survival rates?
\end{enumerate}

\begin{Shaded}
\begin{Highlighting}[]
\FunctionTok{library}\NormalTok{(survival)}
\NormalTok{Churn}\SpecialCharTok{$}\NormalTok{churn1 }\OtherTok{\textless{}{-}} \DecValTok{0}
\NormalTok{Churn}\SpecialCharTok{$}\NormalTok{churn1[Churn}\SpecialCharTok{$}\NormalTok{Churn}\SpecialCharTok{==}\StringTok{"Yes"}\NormalTok{]}\OtherTok{\textless{}{-}}\DecValTok{1}
\NormalTok{survminer}\SpecialCharTok{::}\FunctionTok{ggsurvplot}\NormalTok{(}
    \AttributeTok{fit =}\NormalTok{ survival}\SpecialCharTok{::}\FunctionTok{survfit}\NormalTok{(survival}\SpecialCharTok{::}\FunctionTok{Surv}\NormalTok{(tenure, churn1) }\SpecialCharTok{\textasciitilde{}}\NormalTok{ gender}\SpecialCharTok{+}\NormalTok{SeniorCitizen, }\AttributeTok{data =}\NormalTok{ Churn), }
    \AttributeTok{xlab =} \StringTok{"Days"}\NormalTok{,}
    \AttributeTok{ylab =} \StringTok{"Overall survival probability"}\NormalTok{,}
    \AttributeTok{legend.title =} \StringTok{"Types of People"}\NormalTok{,}
    \AttributeTok{conf.int =} \ConstantTok{TRUE}\NormalTok{,}
    \AttributeTok{legend.labs =} \FunctionTok{c}\NormalTok{(}\StringTok{"Female \& Non Senior"}\NormalTok{, }\StringTok{"Female \& Senior"}\NormalTok{,}\StringTok{"Male \& Non Senior"}\NormalTok{,}\StringTok{"Male \& Senior"}\NormalTok{),}
    \AttributeTok{break.x.by =} \DecValTok{100}\NormalTok{, }
    \AttributeTok{censor =} \ConstantTok{FALSE}\NormalTok{)}
\end{Highlighting}
\end{Shaded}

\includegraphics{ICA5_files/figure-latex/unnamed-chunk-4-1.pdf} Answer:
It appears that the Female \& Non Senior and Male \& Non Senior groups
have higher survival probabilities over time compared to the Female \&
Senior and Male \& Senior groups. This suggests that being a senior
citizen may be associated with lower survival rates, regardless of
gender.

\begin{enumerate}
\def\labelenumi{\arabic{enumi})}
\setcounter{enumi}{4}
\tightlist
\item
  Estimate a Kaplan Meier survival model that uses contract type as the
  explanatory variable. Do we see a difference between contract type?
\end{enumerate}

\begin{Shaded}
\begin{Highlighting}[]
\NormalTok{Churn}\SpecialCharTok{$}\NormalTok{churn1 }\OtherTok{\textless{}{-}} \DecValTok{0}
\NormalTok{Churn}\SpecialCharTok{$}\NormalTok{churn1[Churn}\SpecialCharTok{$}\NormalTok{Churn}\SpecialCharTok{==}\StringTok{"Yes"}\NormalTok{]}\OtherTok{\textless{}{-}}\DecValTok{1}
\NormalTok{survminer}\SpecialCharTok{::}\FunctionTok{ggsurvplot}\NormalTok{(}
    \AttributeTok{fit =}\NormalTok{ survival}\SpecialCharTok{::}\FunctionTok{survfit}\NormalTok{(survival}\SpecialCharTok{::}\FunctionTok{Surv}\NormalTok{(tenure, churn1) }\SpecialCharTok{\textasciitilde{}}\NormalTok{ Contract, }\AttributeTok{data =}\NormalTok{ Churn), }
    \AttributeTok{xlab =} \StringTok{"Days"}\NormalTok{,}
    \AttributeTok{ylab =} \StringTok{"Overall survival probability"}\NormalTok{,}
    \AttributeTok{legend.title =} \StringTok{"Contract Type"}\NormalTok{,}
    \AttributeTok{conf.int =} \ConstantTok{TRUE}\NormalTok{,}
    \AttributeTok{legend.labs =} \FunctionTok{c}\NormalTok{(}\StringTok{"Month to Month"}\NormalTok{, }\StringTok{"One Year"}\NormalTok{,}\StringTok{"Two Year"}\NormalTok{),}
    \AttributeTok{break.x.by =} \DecValTok{100}\NormalTok{, }
    \AttributeTok{censor =} \ConstantTok{FALSE}\NormalTok{)}
\end{Highlighting}
\end{Shaded}

\includegraphics{ICA5_files/figure-latex/unnamed-chunk-5-1.pdf} Answer:
Month to Month, It shows a steep drop, signifying a high probability of
churn in a shorter period. One Year, It presents a moderate fall,
indicating a churn rate that is lower than Month to Month contracts but
higher than Two Year contracts. Two Year: It displays the most gradual
decrease, suggesting the greatest retention rate among the three
contract categories. In conclusion, the chart underscores the
substantial influence of contract duration on the likelihood of churn,
with Two Year contracts correlating with the smallest churn rate,
followed by One Year contracts, and Month to Month contracts exhibiting
the highest churn rate.

\begin{enumerate}
\def\labelenumi{\arabic{enumi})}
\setcounter{enumi}{5}
\tightlist
\item
  Estimate a Cox proportional hazard model of \texttt{tenure}. Use the
  following variables as explanatory variables: gender, seniorcitizen,
  contract type, partner, dependents, type of internet access, do they
  have phone service, and do they use paperless billing.
\end{enumerate}

\begin{Shaded}
\begin{Highlighting}[]
\NormalTok{FIT }\OtherTok{\textless{}{-}}\NormalTok{ survival}\SpecialCharTok{::}\FunctionTok{coxph}\NormalTok{(survival}\SpecialCharTok{::}\FunctionTok{Surv}\NormalTok{(tenure, churn1) }\SpecialCharTok{\textasciitilde{}}\NormalTok{ Contract }\SpecialCharTok{+}\NormalTok{ gender }\SpecialCharTok{+}\NormalTok{ SeniorCitizen }\SpecialCharTok{+}\NormalTok{ Partner }\SpecialCharTok{+}\NormalTok{ Dependents }\SpecialCharTok{+}\NormalTok{ InternetService }\SpecialCharTok{+}\NormalTok{ PhoneService }\SpecialCharTok{+}\NormalTok{ PaperlessBilling, }\AttributeTok{data =}\NormalTok{ Churn)}

\NormalTok{name2\_fit }\OtherTok{\textless{}{-}}\NormalTok{ broom}\SpecialCharTok{::}\FunctionTok{tidy}\NormalTok{(FIT)}
\FunctionTok{colnames}\NormalTok{(name2\_fit) }\OtherTok{\textless{}{-}} \FunctionTok{c}\NormalTok{(}\StringTok{"Term"}\NormalTok{, }\StringTok{"Estimate"}\NormalTok{, }\StringTok{"Standard Error"}\NormalTok{, }\StringTok{"Statistic"}\NormalTok{, }\StringTok{"p value"}\NormalTok{)}
\NormalTok{name2\_fit}\SpecialCharTok{$}\NormalTok{Term }\OtherTok{\textless{}{-}} \FunctionTok{gsub}\NormalTok{(}\StringTok{"ContractOne year"}\NormalTok{, }\StringTok{"Contract= One year"}\NormalTok{, name2\_fit}\SpecialCharTok{$}\NormalTok{Term)}
\NormalTok{name2\_fit}\SpecialCharTok{$}\NormalTok{Term }\OtherTok{\textless{}{-}} \FunctionTok{gsub}\NormalTok{(}\StringTok{"ContractTwo year"}\NormalTok{, }\StringTok{"Contract= Two year"}\NormalTok{, name2\_fit}\SpecialCharTok{$}\NormalTok{Term)}
\NormalTok{name2\_fit}\SpecialCharTok{$}\NormalTok{Term }\OtherTok{\textless{}{-}} \FunctionTok{gsub}\NormalTok{(}\StringTok{"genderMale"}\NormalTok{, }\StringTok{"Gender= Male"}\NormalTok{, name2\_fit}\SpecialCharTok{$}\NormalTok{Term)}
\NormalTok{name2\_fit}\SpecialCharTok{$}\NormalTok{Term }\OtherTok{\textless{}{-}} \FunctionTok{gsub}\NormalTok{(}\StringTok{"SeniorCitizen"}\NormalTok{, }\StringTok{"Senior Citizen"}\NormalTok{, name2\_fit}\SpecialCharTok{$}\NormalTok{Term)}
\NormalTok{name2\_fit}\SpecialCharTok{$}\NormalTok{Term }\OtherTok{\textless{}{-}} \FunctionTok{gsub}\NormalTok{(}\StringTok{"PartnerYes"}\NormalTok{, }\StringTok{"Partner= Yes"}\NormalTok{, name2\_fit}\SpecialCharTok{$}\NormalTok{Term)}
\NormalTok{name2\_fit}\SpecialCharTok{$}\NormalTok{Term }\OtherTok{\textless{}{-}} \FunctionTok{gsub}\NormalTok{(}\StringTok{"DependentsYes"}\NormalTok{, }\StringTok{"Dependents= Yes"}\NormalTok{, name2\_fit}\SpecialCharTok{$}\NormalTok{Term)}
\NormalTok{name2\_fit}\SpecialCharTok{$}\NormalTok{Term }\OtherTok{\textless{}{-}} \FunctionTok{gsub}\NormalTok{(}\StringTok{"InternetServiceFiber optic"}\NormalTok{, }\StringTok{"Internet Service= Fiber optic"}\NormalTok{, name2\_fit}\SpecialCharTok{$}\NormalTok{Term)}
\NormalTok{name2\_fit}\SpecialCharTok{$}\NormalTok{Term }\OtherTok{\textless{}{-}} \FunctionTok{gsub}\NormalTok{(}\StringTok{"InternetServiceNo"}\NormalTok{, }\StringTok{"Internet Service= No"}\NormalTok{, name2\_fit}\SpecialCharTok{$}\NormalTok{Term)}
\NormalTok{name2\_fit}\SpecialCharTok{$}\NormalTok{Term }\OtherTok{\textless{}{-}} \FunctionTok{gsub}\NormalTok{(}\StringTok{"PhoneServiceYes"}\NormalTok{, }\StringTok{"Phone Service= Yes"}\NormalTok{, name2\_fit}\SpecialCharTok{$}\NormalTok{Term)}
\NormalTok{name2\_fit}\SpecialCharTok{$}\NormalTok{Term }\OtherTok{\textless{}{-}} \FunctionTok{gsub}\NormalTok{(}\StringTok{"PaperlessBillingYes"}\NormalTok{, }\StringTok{"Paperless Billing= Yes"}\NormalTok{, name2\_fit}\SpecialCharTok{$}\NormalTok{Term)}

\FunctionTok{print}\NormalTok{(name2\_fit)}
\end{Highlighting}
\end{Shaded}

\begin{verbatim}
## # A tibble: 10 x 5
##    Term                          Estimate `Standard Error` Statistic `p value`
##    <chr>                            <dbl>            <dbl>     <dbl>     <dbl>
##  1 Contract= One year             -2.03             0.0850   -23.9   8.42e-126
##  2 Contract= Two year             -3.90             0.160    -24.4   1.36e-131
##  3 Gender= Male                   -0.0443           0.0463    -0.956 3.39e-  1
##  4 Senior Citizen                 -0.0626           0.0558    -1.12  2.62e-  1
##  5 Partner= Yes                   -0.611            0.0543   -11.3   2.25e- 29
##  6 Dependents= Yes                -0.0946           0.0681    -1.39  1.65e-  1
##  7 Internet Service= Fiber optic   0.360            0.0678     5.32  1.05e-  7
##  8 Internet Service= No           -0.225            0.112     -2.00  4.53e-  2
##  9 Phone Service= Yes             -0.123            0.0974    -1.26  2.08e-  1
## 10 Paperless Billing= Yes          0.123            0.0561     2.20  2.80e-  2
\end{verbatim}

One year lower hazard Males have slightly lower Partner lower

\end{document}
