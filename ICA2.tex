% Options for packages loaded elsewhere
\PassOptionsToPackage{unicode}{hyperref}
\PassOptionsToPackage{hyphens}{url}
%
\documentclass[
]{article}
\usepackage{amsmath,amssymb}
\usepackage{iftex}
\ifPDFTeX
  \usepackage[T1]{fontenc}
  \usepackage[utf8]{inputenc}
  \usepackage{textcomp} % provide euro and other symbols
\else % if luatex or xetex
  \usepackage{unicode-math} % this also loads fontspec
  \defaultfontfeatures{Scale=MatchLowercase}
  \defaultfontfeatures[\rmfamily]{Ligatures=TeX,Scale=1}
\fi
\usepackage{lmodern}
\ifPDFTeX\else
  % xetex/luatex font selection
\fi
% Use upquote if available, for straight quotes in verbatim environments
\IfFileExists{upquote.sty}{\usepackage{upquote}}{}
\IfFileExists{microtype.sty}{% use microtype if available
  \usepackage[]{microtype}
  \UseMicrotypeSet[protrusion]{basicmath} % disable protrusion for tt fonts
}{}
\makeatletter
\@ifundefined{KOMAClassName}{% if non-KOMA class
  \IfFileExists{parskip.sty}{%
    \usepackage{parskip}
  }{% else
    \setlength{\parindent}{0pt}
    \setlength{\parskip}{6pt plus 2pt minus 1pt}}
}{% if KOMA class
  \KOMAoptions{parskip=half}}
\makeatother
\usepackage{xcolor}
\usepackage[margin=1in]{geometry}
\usepackage{color}
\usepackage{fancyvrb}
\newcommand{\VerbBar}{|}
\newcommand{\VERB}{\Verb[commandchars=\\\{\}]}
\DefineVerbatimEnvironment{Highlighting}{Verbatim}{commandchars=\\\{\}}
% Add ',fontsize=\small' for more characters per line
\usepackage{framed}
\definecolor{shadecolor}{RGB}{248,248,248}
\newenvironment{Shaded}{\begin{snugshade}}{\end{snugshade}}
\newcommand{\AlertTok}[1]{\textcolor[rgb]{0.94,0.16,0.16}{#1}}
\newcommand{\AnnotationTok}[1]{\textcolor[rgb]{0.56,0.35,0.01}{\textbf{\textit{#1}}}}
\newcommand{\AttributeTok}[1]{\textcolor[rgb]{0.13,0.29,0.53}{#1}}
\newcommand{\BaseNTok}[1]{\textcolor[rgb]{0.00,0.00,0.81}{#1}}
\newcommand{\BuiltInTok}[1]{#1}
\newcommand{\CharTok}[1]{\textcolor[rgb]{0.31,0.60,0.02}{#1}}
\newcommand{\CommentTok}[1]{\textcolor[rgb]{0.56,0.35,0.01}{\textit{#1}}}
\newcommand{\CommentVarTok}[1]{\textcolor[rgb]{0.56,0.35,0.01}{\textbf{\textit{#1}}}}
\newcommand{\ConstantTok}[1]{\textcolor[rgb]{0.56,0.35,0.01}{#1}}
\newcommand{\ControlFlowTok}[1]{\textcolor[rgb]{0.13,0.29,0.53}{\textbf{#1}}}
\newcommand{\DataTypeTok}[1]{\textcolor[rgb]{0.13,0.29,0.53}{#1}}
\newcommand{\DecValTok}[1]{\textcolor[rgb]{0.00,0.00,0.81}{#1}}
\newcommand{\DocumentationTok}[1]{\textcolor[rgb]{0.56,0.35,0.01}{\textbf{\textit{#1}}}}
\newcommand{\ErrorTok}[1]{\textcolor[rgb]{0.64,0.00,0.00}{\textbf{#1}}}
\newcommand{\ExtensionTok}[1]{#1}
\newcommand{\FloatTok}[1]{\textcolor[rgb]{0.00,0.00,0.81}{#1}}
\newcommand{\FunctionTok}[1]{\textcolor[rgb]{0.13,0.29,0.53}{\textbf{#1}}}
\newcommand{\ImportTok}[1]{#1}
\newcommand{\InformationTok}[1]{\textcolor[rgb]{0.56,0.35,0.01}{\textbf{\textit{#1}}}}
\newcommand{\KeywordTok}[1]{\textcolor[rgb]{0.13,0.29,0.53}{\textbf{#1}}}
\newcommand{\NormalTok}[1]{#1}
\newcommand{\OperatorTok}[1]{\textcolor[rgb]{0.81,0.36,0.00}{\textbf{#1}}}
\newcommand{\OtherTok}[1]{\textcolor[rgb]{0.56,0.35,0.01}{#1}}
\newcommand{\PreprocessorTok}[1]{\textcolor[rgb]{0.56,0.35,0.01}{\textit{#1}}}
\newcommand{\RegionMarkerTok}[1]{#1}
\newcommand{\SpecialCharTok}[1]{\textcolor[rgb]{0.81,0.36,0.00}{\textbf{#1}}}
\newcommand{\SpecialStringTok}[1]{\textcolor[rgb]{0.31,0.60,0.02}{#1}}
\newcommand{\StringTok}[1]{\textcolor[rgb]{0.31,0.60,0.02}{#1}}
\newcommand{\VariableTok}[1]{\textcolor[rgb]{0.00,0.00,0.00}{#1}}
\newcommand{\VerbatimStringTok}[1]{\textcolor[rgb]{0.31,0.60,0.02}{#1}}
\newcommand{\WarningTok}[1]{\textcolor[rgb]{0.56,0.35,0.01}{\textbf{\textit{#1}}}}
\usepackage{graphicx}
\makeatletter
\def\maxwidth{\ifdim\Gin@nat@width>\linewidth\linewidth\else\Gin@nat@width\fi}
\def\maxheight{\ifdim\Gin@nat@height>\textheight\textheight\else\Gin@nat@height\fi}
\makeatother
% Scale images if necessary, so that they will not overflow the page
% margins by default, and it is still possible to overwrite the defaults
% using explicit options in \includegraphics[width, height, ...]{}
\setkeys{Gin}{width=\maxwidth,height=\maxheight,keepaspectratio}
% Set default figure placement to htbp
\makeatletter
\def\fps@figure{htbp}
\makeatother
\setlength{\emergencystretch}{3em} % prevent overfull lines
\providecommand{\tightlist}{%
  \setlength{\itemsep}{0pt}\setlength{\parskip}{0pt}}
\setcounter{secnumdepth}{-\maxdimen} % remove section numbering
\usepackage{booktabs}
\usepackage{longtable}
\usepackage{array}
\usepackage{multirow}
\usepackage{wrapfig}
\usepackage{float}
\usepackage{colortbl}
\usepackage{pdflscape}
\usepackage{tabu}
\usepackage{threeparttable}
\usepackage{threeparttablex}
\usepackage[normalem]{ulem}
\usepackage{makecell}
\usepackage{xcolor}
\usepackage{siunitx}

  \newcolumntype{d}{S[
    input-open-uncertainty=,
    input-close-uncertainty=,
    parse-numbers = false,
    table-align-text-pre=false,
    table-align-text-post=false
  ]}
  
\ifLuaTeX
  \usepackage{selnolig}  % disable illegal ligatures
\fi
\IfFileExists{bookmark.sty}{\usepackage{bookmark}}{\usepackage{hyperref}}
\IfFileExists{xurl.sty}{\usepackage{xurl}}{} % add URL line breaks if available
\urlstyle{same}
\hypersetup{
  pdftitle={ICA 2},
  pdfauthor={Alice Trejo},
  hidelinks,
  pdfcreator={LaTeX via pandoc}}

\title{ICA 2}
\author{Alice Trejo}
\date{MSBA Data Analytics III}

\begin{document}
\maketitle

\hypertarget{learning-objectives}{%
\subsection{Learning Objectives}\label{learning-objectives}}

In this assignment, you will practice your T-Test and R programming
skills. You will use a fixed effects model and perform a
difference-in-differences analysis. Please answer the questions in a R
markdown file and ``Knit'' the file so that I can see your analysis.

\hypertarget{experiments}{%
\subsection{Experiments}\label{experiments}}

Reanalysis of Gerber, Green and Larimer (2008) `Why do large numbers of
people vote, despite the fact that, as Hegel once observed, ``the
casting of a single vote is of no significance where there is a
multitude of electors''?'

This is the question that drives the experimental analysis of Gerber,
Green and Larimer (2008). If it is irrational to vote because the costs
of doings so (time spent informing oneself, time spent getting to the
polling station, etc) are clearly greater than the gains to be made from
voting (the probability that any individual voter will be decisive in an
election are vanishingly small), then why do we observe millions of
people voting in elections? One commonly proposed answer is that voters
may have some sense of civic duty which drives them to the polls.
Gerber, Green and Larimer investigate this idea empirically by priming
voters to think about civic duty while also varying the amount of social
pressure voters are subject to. In a field experiment in advance of the
2006 primary election in Michigan, nearly 350,000 voters were assigned
at random to one of four treatment groups, where voters received
mailouts which encouraged them to vote, or a control group where voters
received no mailout. The treatment and control conditions were as
follows:

\begin{verbatim}
Treatment 1 (“Civic duty”): Voters receive mailout reminding them that voting is a civic duty
Treatment 2 (“Hawthorne”): Voters receive mailout telling them that researchers would be studying their turnout based on public records
Treatment 3 (“Self”): Voters receive mailout displaying the record of turnout for their household in prior elections.
Treatment 4 (“Neighbors”): Voters receive mailout displaying the record of turnout for their household and their neighbours’ households in prior elections.
Control: Voters receive no mailout.
\end{verbatim}

Load the replication data for Gerber, Green and Larimer (2008). This
data is stored in a .Rdata format, which is the main way to save data in
R. Therefore you will not be able to use read.csv but instead should use
the function load.

\begin{Shaded}
\begin{Highlighting}[]
\CommentTok{\# You will need to change the file location for the code to work.}
\FunctionTok{load}\NormalTok{(}\StringTok{"gerber\_green\_larimer.Rdata"}\NormalTok{)}
\end{Highlighting}
\end{Shaded}

Once you have loaded the data, familiarise yourself with the the gerber
object which should be in your current envionment. Use the str and
summary functions to get an idea of what is in the data. There are 5
variables in this data.frame: Variable name Description

\begin{itemize}
\tightlist
\item
  voted: Indicator for whether the voter voted in the 2006 election (1)
  or did not vote (0)
\item
  treatment: Factor variable indicating which treatment arm (or control
  group) the voter was allocated to
\item
  sex: Sex of the respondent
\item
  yob: Year of birth of the respondent
\item
  p2004: Indicator for whether the voter voted in the 2004 election
  (Yes) or not (No)
\end{itemize}

\begin{enumerate}
\def\labelenumi{\arabic{enumi}.}
\tightlist
\item
  Calculate the turnout rates, ``voted'', for each of the experimental
  groups (4 treatments, 1 control). Calculate the number of individuals
  allocated to each group. Recreate table 2 on p.~38 of the paper.
\end{enumerate}

\begin{Shaded}
\begin{Highlighting}[]
\FunctionTok{library}\NormalTok{(modelsummary)}
\end{Highlighting}
\end{Shaded}

\begin{verbatim}
## Version 2.0.0 of `modelsummary`, to be released soon, will introduce a
##   breaking change: The default table-drawing package will be `tinytable`
##   instead of `kableExtra`. All currently supported table-drawing packages
##   will continue to be supported for the foreseeable future, including
##   `kableExtra`, `gt`, `huxtable`, `flextable, and `DT`.
##   
##   You can always call the `config_modelsummary()` function to change the
##   default table-drawing package in persistent fashion. To try `tinytable`
##   now:
##   
##   config_modelsummary(factory_default = 'tinytable')
##   
##   To set the default back to `kableExtra`:
##   
##   config_modelsummary(factory_default = 'kableExtra')
\end{verbatim}

\begin{Shaded}
\begin{Highlighting}[]
\FunctionTok{datasummary}\NormalTok{(voted }\SpecialCharTok{\textasciitilde{}}\NormalTok{ Mean}\SpecialCharTok{*}\FunctionTok{factor}\NormalTok{(treatment), }\AttributeTok{data=}\NormalTok{gerber)}
\end{Highlighting}
\end{Shaded}

\begin{table}
\centering
\begin{tabular}[t]{lrrrrr}
\toprule
  & Control & Civic Duty & Hawthorne & Self & Neighbors\\
\midrule
voted & \num{0.30} & \num{0.31} & \num{0.32} & \num{0.35} & \num{0.38}\\
\bottomrule
\end{tabular}
\end{table}

\begin{enumerate}
\def\labelenumi{\arabic{enumi}.}
\setcounter{enumi}{1}
\tightlist
\item
  Use the following code to create three new variables in the
  data.frame. First, a variable that is equal to 1 if a respondent is
  female, and 0 otherwise. Second, a variable that measures the age of
  each voter in years at the time of the experiment (which was conducted
  in 2006). Third, a variable that is equal to 1 if the voter voted in
  the November 2004 Miderm election.
\end{enumerate}

\begin{Shaded}
\begin{Highlighting}[]
\DocumentationTok{\#\# Female dummy variable}
\NormalTok{gerber}\SpecialCharTok{$}\NormalTok{female }\OtherTok{\textless{}{-}} \FunctionTok{ifelse}\NormalTok{(gerber}\SpecialCharTok{$}\NormalTok{sex }\SpecialCharTok{==} \StringTok{"female"}\NormalTok{, }\DecValTok{1}\NormalTok{, }\DecValTok{0}\NormalTok{)}

\DocumentationTok{\#\# Age variable}
\NormalTok{gerber}\SpecialCharTok{$}\NormalTok{age }\OtherTok{\textless{}{-}} \DecValTok{2006} \SpecialCharTok{{-}}\NormalTok{ gerber}\SpecialCharTok{$}\NormalTok{yob}

\DocumentationTok{\#\# 2004 variable}
\NormalTok{gerber}\SpecialCharTok{$}\NormalTok{turnout04 }\OtherTok{\textless{}{-}} \FunctionTok{ifelse}\NormalTok{(gerber}\SpecialCharTok{$}\NormalTok{p2004 }\SpecialCharTok{==} \StringTok{"Yes"}\NormalTok{, }\DecValTok{1}\NormalTok{, }\DecValTok{0}\NormalTok{)}
\end{Highlighting}
\end{Shaded}

\begin{enumerate}
\def\labelenumi{\arabic{enumi}.}
\setcounter{enumi}{2}
\tightlist
\item
  Using these variables, conduct balance checks to establish whether
  there are potentially confounding differences between treatment and
  control groups. You do this by using the variables female, age, and
  turnout04 as depedent variables. Use just the factor variable of
  treatment as your explanatory variable. Can you conclude from the
  results that randomization worked? How do you know?
\end{enumerate}

\begin{Shaded}
\begin{Highlighting}[]
\NormalTok{exp1 }\OtherTok{\textless{}{-}}\FunctionTok{lm}\NormalTok{(female }\SpecialCharTok{\textasciitilde{}} \FunctionTok{factor}\NormalTok{(treatment), }\AttributeTok{data =}\NormalTok{gerber)}
\NormalTok{exp2 }\OtherTok{\textless{}{-}}\FunctionTok{lm}\NormalTok{(age }\SpecialCharTok{\textasciitilde{}} \FunctionTok{factor}\NormalTok{(treatment), }\AttributeTok{data =}\NormalTok{gerber)}
\NormalTok{exp3 }\OtherTok{\textless{}{-}}\FunctionTok{lm}\NormalTok{(turnout04 }\SpecialCharTok{\textasciitilde{}} \FunctionTok{factor}\NormalTok{(treatment), }\AttributeTok{data =}\NormalTok{gerber)}
\FunctionTok{library}\NormalTok{(modelsummary)}
\FunctionTok{modelsummary}\NormalTok{(}\FunctionTok{list}\NormalTok{(}\StringTok{"Female"}\OtherTok{=}\NormalTok{exp1,}\StringTok{"Age"}\OtherTok{=}\NormalTok{exp2,}\StringTok{"Turnout"}\OtherTok{=}\NormalTok{exp3),}\AttributeTok{coef\_rename =}\NormalTok{ coef\_rename, }\AttributeTok{stars =} \ConstantTok{TRUE}\NormalTok{)}
\end{Highlighting}
\end{Shaded}

\begin{table}
\centering
\begin{tabular}[t]{lccc}
\toprule
  & Female & Age & Turnout\\
\midrule
(Intercept) & \num{0.499}*** & \num{49.814}*** & \num{0.400}***\\
 & (\num{0.001}) & (\num{0.033}) & (\num{0.001})\\
Civic Duty & \num{0.001} & \num{-0.155}+ & \num{-0.001}\\
 & (\num{0.003}) & (\num{0.081}) & \vphantom{3} (\num{0.003})\\
Hawthorne & \num{0.000} & \num{-0.109} & \num{0.003}\\
 & (\num{0.003}) & (\num{0.081}) & \vphantom{2} (\num{0.003})\\
Self & \num{0.001} & \num{-0.021} & \num{0.002}\\
 & (\num{0.003}) & (\num{0.081}) & \vphantom{1} (\num{0.003})\\
Neighbors & \num{0.001} & \num{0.039} & \num{0.006}*\\
 & (\num{0.003}) & (\num{0.081}) & (\num{0.003})\\
\midrule
Num.Obs. & \num{344084} & \num{344084} & \num{344084}\\
R2 & \num{0.000} & \num{0.000} & \num{0.000}\\
R2 Adj. & \num{0.000} & \num{0.000} & \num{0.000}\\
AIC & \num{499477.3} & \num{2814316.6} & \num{485852.4}\\
BIC & \num{499541.8} & \num{2814381.1} & \num{485916.9}\\
Log.Lik. & \num{-249732.671} & \num{-1407152.317} & \num{-242920.207}\\
RMSE & \num{0.50} & \num{14.45} & \num{0.49}\\
\bottomrule
\multicolumn{4}{l}{\rule{0pt}{1em}+ p $<$ 0.1, * p $<$ 0.05, ** p $<$ 0.01, *** p $<$ 0.001}\\
\end{tabular}
\end{table}

Answer: We can conclude from the results that randomization worked
because there are not relevant coefficients and R2 is 0 in this case.

\begin{enumerate}
\def\labelenumi{\arabic{enumi}.}
\setcounter{enumi}{3}
\tightlist
\item
  Estimate the average treatment effects of the different treatment arms
  whilst controlling for the variables you created for the question
  above. How do these estimates differ from regression estimates of the
  treatment effects only (i.e.~without controlling for other factors)?
  Why?
\end{enumerate}

\begin{Shaded}
\begin{Highlighting}[]
\NormalTok{exp4 }\OtherTok{\textless{}{-}}\FunctionTok{lm}\NormalTok{(voted }\SpecialCharTok{\textasciitilde{}} \FunctionTok{factor}\NormalTok{(treatment), }\AttributeTok{data =}\NormalTok{gerber)}
\CommentTok{\# Now use the same equation above but add female, age, and turnout04}
\NormalTok{exp5 }\OtherTok{\textless{}{-}}\FunctionTok{lm}\NormalTok{(voted }\SpecialCharTok{\textasciitilde{}} \FunctionTok{factor}\NormalTok{(treatment)}\SpecialCharTok{+}\NormalTok{female}\SpecialCharTok{+}\NormalTok{age}\SpecialCharTok{+}\NormalTok{turnout04, }\AttributeTok{data=}\NormalTok{gerber)}
\FunctionTok{modelsummary}\NormalTok{(}\FunctionTok{list}\NormalTok{(exp4,exp5),}\AttributeTok{coef\_rename =}\NormalTok{ coef\_rename, }\AttributeTok{stars =} \ConstantTok{TRUE}\NormalTok{)}
\end{Highlighting}
\end{Shaded}

\begin{table}
\centering
\begin{tabular}[t]{lcc}
\toprule
  & (1) & (2)\\
\midrule
(Intercept) & \num{0.297}*** & \num{0.044}***\\
 & (\num{0.001}) & (\num{0.003})\\
Civic Duty & \num{0.018}*** & \num{0.019}***\\
 & (\num{0.003}) & \vphantom{3} (\num{0.003})\\
Hawthorne & \num{0.026}*** & \num{0.026}***\\
 & (\num{0.003}) & \vphantom{2} (\num{0.003})\\
Self & \num{0.049}*** & \num{0.048}***\\
 & (\num{0.003}) & \vphantom{1} (\num{0.003})\\
Neighbors & \num{0.081}*** & \num{0.080}***\\
 & (\num{0.003}) & (\num{0.003})\\
Female &  & \num{-0.008}***\\
 &  & \vphantom{1} (\num{0.002})\\
Age &  & \num{0.004}***\\
 &  & (\num{0.000})\\
Turnout04 &  & \num{0.148}***\\
 &  & (\num{0.002})\\
\midrule
Num.Obs. & \num{344084} & \num{344084}\\
R2 & \num{0.003} & \num{0.045}\\
R2 Adj. & \num{0.003} & \num{0.045}\\
AIC & \num{448179.9} & \num{433501.1}\\
BIC & \num{448244.4} & \num{433597.9}\\
Log.Lik. & \num{-224083.935} & \num{-216741.564}\\
RMSE & \num{0.46} & \num{0.45}\\
\bottomrule
\multicolumn{3}{l}{\rule{0pt}{1em}+ p $<$ 0.1, * p $<$ 0.05, ** p $<$ 0.01, *** p $<$ 0.001}\\
\end{tabular}
\end{table}

Answer: The coefficients are almost the same or the same, because the
randomization worked. If we add truly independent variables it shouldn't
change our coefficients.

\begin{enumerate}
\def\labelenumi{\arabic{enumi}.}
\setcounter{enumi}{4}
\tightlist
\item
  Estimate the treatment effects separately for men and women. Do you
  note any differences in the impact of the treatment among these
  subgroups? Answer: There are not big differences.
\end{enumerate}

\begin{Shaded}
\begin{Highlighting}[]
\CommentTok{\# modify the equation below for just men}
\NormalTok{exp6 }\OtherTok{\textless{}{-}}\FunctionTok{lm}\NormalTok{(voted }\SpecialCharTok{\textasciitilde{}} \FunctionTok{factor}\NormalTok{(treatment), }\AttributeTok{data =}\NormalTok{gerber[gerber}\SpecialCharTok{$}\NormalTok{female}\SpecialCharTok{==}\DecValTok{0}\NormalTok{, ])}
\CommentTok{\# modify the equation below for just women}
\NormalTok{exp7 }\OtherTok{\textless{}{-}}\FunctionTok{lm}\NormalTok{(voted }\SpecialCharTok{\textasciitilde{}} \FunctionTok{factor}\NormalTok{(treatment), }\AttributeTok{data =}\NormalTok{gerber[gerber}\SpecialCharTok{$}\NormalTok{female}\SpecialCharTok{==}\DecValTok{1}\NormalTok{, ])}
\FunctionTok{modelsummary}\NormalTok{(}\FunctionTok{list}\NormalTok{(}\StringTok{"Men"}\OtherTok{=}\NormalTok{exp6,}\StringTok{"Women"}\OtherTok{=}\NormalTok{exp7),}\AttributeTok{coef\_rename =}\NormalTok{ coef\_rename, }\AttributeTok{stars =} \ConstantTok{TRUE}\NormalTok{)}
\end{Highlighting}
\end{Shaded}

\begin{table}
\centering
\begin{tabular}[t]{lcc}
\toprule
  & Men & Women\\
\midrule
(Intercept) & \num{0.303}*** & \num{0.290}***\\
 & (\num{0.002}) & (\num{0.001})\\
Civic Duty & \num{0.020}*** & \num{0.016}***\\
 & (\num{0.004}) & \vphantom{3} (\num{0.004})\\
Hawthorne & \num{0.025}*** & \num{0.027}***\\
 & (\num{0.004}) & \vphantom{2} (\num{0.004})\\
Self & \num{0.046}*** & \num{0.051}***\\
 & (\num{0.004}) & \vphantom{1} (\num{0.004})\\
Neighbors & \num{0.082}*** & \num{0.081}***\\
 & (\num{0.004}) & (\num{0.004})\\
\midrule
Num.Obs. & \num{172289} & \num{171795}\\
R2 & \num{0.003} & \num{0.004}\\
R2 Adj. & \num{0.003} & \num{0.003}\\
AIC & \num{226155.0} & \num{221958.4}\\
BIC & \num{226215.3} & \num{222018.8}\\
Log.Lik. & \num{-113071.476} & \num{-110973.214}\\
RMSE & \num{0.47} & \num{0.46}\\
\bottomrule
\multicolumn{3}{l}{\rule{0pt}{1em}+ p $<$ 0.1, * p $<$ 0.05, ** p $<$ 0.01, *** p $<$ 0.001}\\
\end{tabular}
\end{table}

Difference-in-Differences: Replication Exercise from the notes

The data are about the expansion of the Earned Income Tax Credit. The
sample only contains single women. This legislation is aimed at
providing a tax break for low income individuals. For some background on
the subject, see

Eissa, Nada, and Jeffrey B. Liebman. 1996. Labor Supply Responses to the
Earned Income Tax Credit. Quarterly Journal of Economics. 111(2):
605-637.

\textbf{Big Hint: Most of the code you need is in the notes}

Variable Names and Definitions

state: Factor variable containg the state's FIPS code. year: Calendar
Year urate: unemployment rate for the state and year children: number of
children in the household nonwhite: the person identifies as non-White
finc: Family household income earn: Earned income unearn: unearned
income age: Age of the mother in years ed: Years of schooling work:
Indicator variable equal to 1 if the person is currently working

The homework questions:

\begin{enumerate}
\def\labelenumi{\arabic{enumi}.}
\tightlist
\item
  Provide Descriptive Statistics for the data. Format nicely, not just R
  printout. Here is an example below. I have already provided the code
  to read in the data below. You need to create the data summary table.
\end{enumerate}

\begin{Shaded}
\begin{Highlighting}[]
\FunctionTok{require}\NormalTok{(foreign)}
\NormalTok{eitc}\OtherTok{\textless{}{-}}\FunctionTok{read.dta}\NormalTok{(}\StringTok{"https://github.com/CausalReinforcer/Stata/raw/master/eitc.dta"}\NormalTok{)}
\FunctionTok{library}\NormalTok{(modelsummary)}
\CommentTok{\# the data mtcars is just an example. You need to replace it with eitc}
\FunctionTok{datasummary}\NormalTok{((}\StringTok{\textasciigrave{}}\AttributeTok{Unemployment Rate}\StringTok{\textasciigrave{}}\OtherTok{=}\NormalTok{urate)}\SpecialCharTok{+}\NormalTok{children}\SpecialCharTok{+}\NormalTok{(}\StringTok{\textasciigrave{}}\AttributeTok{Non{-}white}\StringTok{\textasciigrave{}}\OtherTok{=}\NormalTok{nonwhite)}\SpecialCharTok{+}\NormalTok{(}\StringTok{\textasciigrave{}}\AttributeTok{Family Income}\StringTok{\textasciigrave{}}\OtherTok{=}\NormalTok{finc)}\SpecialCharTok{+}\NormalTok{(}\StringTok{\textasciigrave{}}\AttributeTok{Earned Income}\StringTok{\textasciigrave{}}\OtherTok{=}\NormalTok{earn)}\SpecialCharTok{+}\NormalTok{age}\SpecialCharTok{+}\NormalTok{(}\StringTok{\textasciigrave{}}\AttributeTok{Years of Education}\StringTok{\textasciigrave{}}\OtherTok{=}\NormalTok{ed)}\SpecialCharTok{+}\NormalTok{work}\SpecialCharTok{+}\NormalTok{(}\StringTok{\textasciigrave{}}\AttributeTok{Unearned Income}\StringTok{\textasciigrave{}}\OtherTok{=}\NormalTok{unearn) }\SpecialCharTok{\textasciitilde{}}\NormalTok{ Mean }\SpecialCharTok{+}\NormalTok{ SD }\SpecialCharTok{+}\NormalTok{ Min }\SpecialCharTok{+}\NormalTok{ Max,}
            \AttributeTok{data =}\NormalTok{ eitc)}
\end{Highlighting}
\end{Shaded}

\begin{table}
\centering
\begin{tabular}[t]{lrrrr}
\toprule
  & Mean & SD & Min & Max\\
\midrule
Unemployment Rate & \num{6.76} & \num{1.46} & \num{2.60} & \num{11.40}\\
children & \num{1.19} & \num{1.38} & \num{0.00} & \num{9.00}\\
Non-white & \num{0.60} & \num{0.49} & \num{0.00} & \num{1.00}\\
Family Income & \num{15255.32} & \num{19444.25} & \num{0.00} & \num{575616.82}\\
Earned Income & \num{10432.48} & \num{18200.76} & \num{0.00} & \num{537880.61}\\
age & \num{35.21} & \num{10.16} & \num{20.00} & \num{54.00}\\
Years of Education & \num{8.81} & \num{2.64} & \num{0.00} & \num{11.00}\\
work & \num{0.51} & \num{0.50} & \num{0.00} & \num{1.00}\\
Unearned Income & \num{4.82} & \num{7.12} & \num{0.00} & \num{134.06}\\
\bottomrule
\end{tabular}
\end{table}

\begin{enumerate}
\def\labelenumi{\arabic{enumi}.}
\setcounter{enumi}{1}
\tightlist
\item
  Calculate the sample means of all variables for (a) single women with
  no children, (b) single women with 1 child, and (c) single women with
  2+ children. \textbf{Hint: Use the tidyverse to make this table. You
  can either filter the data or use dplyr to construct groups.You can
  even use datasummary to do this step. Below is one example}
\end{enumerate}

\begin{Shaded}
\begin{Highlighting}[]
\CommentTok{\# Make the appropriate changes (i.e. dataframe name and correct factor variable)}
\NormalTok{eitc}\SpecialCharTok{$}\NormalTok{nochild}\OtherTok{\textless{}{-}}\NormalTok{eitc}\SpecialCharTok{$}\NormalTok{children}
\NormalTok{eitc}\SpecialCharTok{$}\NormalTok{nochild[eitc}\SpecialCharTok{$}\NormalTok{children}\SpecialCharTok{\textgreater{}}\DecValTok{2}\NormalTok{] }\OtherTok{\textless{}{-}}\DecValTok{2}
\NormalTok{eitc}\SpecialCharTok{$}\NormalTok{nochild }\OtherTok{\textless{}{-}} \FunctionTok{factor}\NormalTok{(eitc}\SpecialCharTok{$}\NormalTok{nochild, }\AttributeTok{labels =} \FunctionTok{c}\NormalTok{(}\StringTok{"No Children"}\NormalTok{, }\StringTok{"1 Child"}\NormalTok{, }\StringTok{"2 or more Children"}\NormalTok{))}
\FunctionTok{datasummary}\NormalTok{((}\StringTok{\textasciigrave{}}\AttributeTok{Unemployment Rate}\StringTok{\textasciigrave{}}\OtherTok{=}\NormalTok{urate)}\SpecialCharTok{+}\NormalTok{children}\SpecialCharTok{+}\NormalTok{(}\StringTok{\textasciigrave{}}\AttributeTok{Non{-}white}\StringTok{\textasciigrave{}}\OtherTok{=}\NormalTok{nonwhite)}\SpecialCharTok{+}\NormalTok{(}\StringTok{\textasciigrave{}}\AttributeTok{Family Income}\StringTok{\textasciigrave{}}\OtherTok{=}\NormalTok{finc)}\SpecialCharTok{+}\NormalTok{(}\StringTok{\textasciigrave{}}\AttributeTok{Earned Income}\StringTok{\textasciigrave{}}\OtherTok{=}\NormalTok{earn)}\SpecialCharTok{+}\NormalTok{age}\SpecialCharTok{+}\NormalTok{(}\StringTok{\textasciigrave{}}\AttributeTok{Years of Education}\StringTok{\textasciigrave{}}\OtherTok{=}\NormalTok{ed)}\SpecialCharTok{+}\NormalTok{work}\SpecialCharTok{+}\NormalTok{(}\StringTok{\textasciigrave{}}\AttributeTok{Unearned Income}\StringTok{\textasciigrave{}}\OtherTok{=}\NormalTok{unearn) }\SpecialCharTok{\textasciitilde{}}\NormalTok{ mean }\SpecialCharTok{*} \FunctionTok{factor}\NormalTok{(nochild),}\AttributeTok{data =}\NormalTok{ eitc)}
\end{Highlighting}
\end{Shaded}

\begin{table}
\centering
\begin{tabular}[t]{lrrr}
\toprule
  & No Children & 1 Child & 2 or more Children\\
\midrule
Unemployment Rate & \num{6.66} & \num{6.80} & \num{6.86}\\
children & \num{0.00} & \num{1.00} & \num{2.80}\\
Non-white & \num{0.52} & \num{0.60} & \num{0.71}\\
Family Income & \num{18559.86} & \num{13941.57} & \num{11985.30}\\
Earned Income & \num{13760.26} & \num{9928.28} & \num{6613.55}\\
age & \num{38.50} & \num{33.76} & \num{32.05}\\
Years of Education & \num{8.55} & \num{8.99} & \num{9.01}\\
work & \num{0.57} & \num{0.54} & \num{0.42}\\
Unearned Income & \num{4.80} & \num{4.01} & \num{5.37}\\
\bottomrule
\end{tabular}
\end{table}

\begin{enumerate}
\def\labelenumi{\arabic{enumi}.}
\setcounter{enumi}{2}
\tightlist
\item
  Construct a variable for the ``treatment'' called ANYKIDS. This
  variable should equal 1 if they have any children and zero otherwise.
  Create a second variable to indicate after the expansion (called
  POST93-should be 1 for 1994 and later).
\end{enumerate}

\begin{Shaded}
\begin{Highlighting}[]
\CommentTok{\# the EITC went into effect in the year 1994}
\NormalTok{eitc}\SpecialCharTok{$}\NormalTok{post93 }\OtherTok{=} \FunctionTok{as.numeric}\NormalTok{(eitc}\SpecialCharTok{$}\NormalTok{year }\SpecialCharTok{\textgreater{}=} \DecValTok{1994}\NormalTok{)}
\CommentTok{\# The EITC only affects women with at least one child, so the}
\CommentTok{\# treatment group will be all women with children.}
\NormalTok{eitc}\SpecialCharTok{$}\NormalTok{anykids }\OtherTok{=} \FunctionTok{as.numeric}\NormalTok{(eitc}\SpecialCharTok{$}\NormalTok{children }\SpecialCharTok{\textgreater{}=} \DecValTok{1}\NormalTok{)}
\end{Highlighting}
\end{Shaded}

\begin{enumerate}
\def\labelenumi{\arabic{enumi}.}
\setcounter{enumi}{3}
\tightlist
\item
  Create a graph which plots mean annual employment rates by year
  (1991-1996) for single women with children (treatment) and without
  children (control). \textbf{Hint: you should have two lines on the
  same graph.} I would suggest to use ggplot to make this plot. Here is
  some sample code. The variable ``work'' is your dependent variable.
\end{enumerate}

\begin{Shaded}
\begin{Highlighting}[]
\NormalTok{minfo }\OtherTok{=} \FunctionTok{aggregate}\NormalTok{(eitc}\SpecialCharTok{$}\NormalTok{work, }\FunctionTok{list}\NormalTok{(eitc}\SpecialCharTok{$}\NormalTok{year,eitc}\SpecialCharTok{$}\NormalTok{anykids }\SpecialCharTok{==} \DecValTok{1}\NormalTok{), mean)}
\CommentTok{\# rename column headings (variables)}
\FunctionTok{names}\NormalTok{(minfo) }\OtherTok{=} \FunctionTok{c}\NormalTok{(}\StringTok{"YR"}\NormalTok{,}\StringTok{"Treatment"}\NormalTok{,}\StringTok{"LFPR"}\NormalTok{)}
\CommentTok{\# Attach a new column with labels}
\NormalTok{minfo}\SpecialCharTok{$}\NormalTok{Group[}\DecValTok{1}\SpecialCharTok{:}\DecValTok{6}\NormalTok{] }\OtherTok{=} \StringTok{"Single women, no children"}
\NormalTok{minfo}\SpecialCharTok{$}\NormalTok{Group[}\DecValTok{7}\SpecialCharTok{:}\DecValTok{12}\NormalTok{] }\OtherTok{=} \StringTok{"Single women, children"}
\CommentTok{\#minfo}
\FunctionTok{require}\NormalTok{(ggplot2)    }\CommentTok{\#package for creating nice plots}
\end{Highlighting}
\end{Shaded}

\begin{verbatim}
## Loading required package: ggplot2
\end{verbatim}

\begin{Shaded}
\begin{Highlighting}[]
\FunctionTok{qplot}\NormalTok{(YR, LFPR, }\AttributeTok{data=}\NormalTok{minfo, }\AttributeTok{geom=}\FunctionTok{c}\NormalTok{(}\StringTok{"point"}\NormalTok{,}\StringTok{"line"}\NormalTok{), }\AttributeTok{colour=}\NormalTok{Group,}
\AttributeTok{xlab=}\StringTok{"Year"}\NormalTok{, }\AttributeTok{ylab=}\StringTok{"Labor Force Participation Rate"}\NormalTok{)}\SpecialCharTok{+}\FunctionTok{geom\_vline}\NormalTok{(}\AttributeTok{xintercept =} \DecValTok{1994}\NormalTok{)}
\end{Highlighting}
\end{Shaded}

\begin{verbatim}
## Warning: `qplot()` was deprecated in ggplot2 3.4.0.
## This warning is displayed once every 8 hours.
## Call `lifecycle::last_lifecycle_warnings()` to see where this warning was
## generated.
\end{verbatim}

\includegraphics{ICA2_files/figure-latex/unnamed-chunk-10-1.pdf}

\begin{enumerate}
\def\labelenumi{\arabic{enumi}.}
\setcounter{enumi}{4}
\item
  Do the trends between the two groups appear to be parallel? Why is
  this important? Answer: Yes, they appear to be parallel until the year
  1994. This is important because it indicates that these both groups
  (Treatment and Control) are being influenced similarly by the overall
  economic conditions, society norms, or employment policies in place
  during that time (1991-1994).
\item
  Calculate the unconditional difference-in-difference estimates of the
  effect of the 1993 EITC expansion on employment of single women.
  \textbf{Hint: This means calculate the DID treatment effect by just
  subtracting means (i.e.~no regression)}
\end{enumerate}

\begin{Shaded}
\begin{Highlighting}[]
\CommentTok{\# Compute the four data points needed in the DID calculation:}
\NormalTok{a }\OtherTok{=} \FunctionTok{sapply}\NormalTok{(}\FunctionTok{subset}\NormalTok{(eitc, post93 }\SpecialCharTok{==} \DecValTok{0} \SpecialCharTok{\&}\NormalTok{ anykids }\SpecialCharTok{==} \DecValTok{0}\NormalTok{, }\AttributeTok{select=}\NormalTok{work), mean)}
\NormalTok{b }\OtherTok{=} \FunctionTok{sapply}\NormalTok{(}\FunctionTok{subset}\NormalTok{(eitc, post93 }\SpecialCharTok{==} \DecValTok{0} \SpecialCharTok{\&}\NormalTok{ anykids }\SpecialCharTok{==} \DecValTok{1}\NormalTok{, }\AttributeTok{select=}\NormalTok{work), mean)}
\NormalTok{c }\OtherTok{=} \FunctionTok{sapply}\NormalTok{(}\FunctionTok{subset}\NormalTok{(eitc, post93 }\SpecialCharTok{==} \DecValTok{1} \SpecialCharTok{\&}\NormalTok{ anykids }\SpecialCharTok{==} \DecValTok{0}\NormalTok{, }\AttributeTok{select=}\NormalTok{work), mean)}
\NormalTok{d }\OtherTok{=} \FunctionTok{sapply}\NormalTok{(}\FunctionTok{subset}\NormalTok{(eitc, post93 }\SpecialCharTok{==} \DecValTok{1} \SpecialCharTok{\&}\NormalTok{ anykids }\SpecialCharTok{==} \DecValTok{1}\NormalTok{, }\AttributeTok{select=}\NormalTok{work), mean)}
\CommentTok{\# Compute the effect of the EITC on the employment of women with children:}
\NormalTok{(d}\SpecialCharTok{{-}}\NormalTok{c)}\SpecialCharTok{{-}}\NormalTok{(b}\SpecialCharTok{{-}}\NormalTok{a)}
\end{Highlighting}
\end{Shaded}

\begin{verbatim}
##       work 
## 0.04687313
\end{verbatim}

\begin{enumerate}
\def\labelenumi{\arabic{enumi}.}
\setcounter{enumi}{6}
\tightlist
\item
  Now run a regression to estimate the conditional
  difference-in-difference estimate of the effect of the EITC. Use all
  women with children as the treatment group. \textbf{Hint: your answers
  for 6 and 7 should match.}
\end{enumerate}

\begin{Shaded}
\begin{Highlighting}[]
\CommentTok{\# Estimate a difference in difference regression. You should be using ANYKIDS and POST93 in your regression. Work is your dependent variable}

\NormalTok{reg1 }\OtherTok{\textless{}{-}} \FunctionTok{lm}\NormalTok{(work }\SpecialCharTok{\textasciitilde{}}\NormalTok{ post93}\SpecialCharTok{+}\NormalTok{anykids}\SpecialCharTok{+}\NormalTok{post93}\SpecialCharTok{*}\NormalTok{anykids, }\AttributeTok{data =}\NormalTok{ eitc)}
\FunctionTok{summary}\NormalTok{(reg1)}
\end{Highlighting}
\end{Shaded}

\begin{verbatim}
## 
## Call:
## lm(formula = work ~ post93 + anykids + post93 * anykids, data = eitc)
## 
## Residuals:
##     Min      1Q  Median      3Q     Max 
## -0.5755 -0.4908  0.4245  0.5092  0.5540 
## 
## Coefficients:
##                 Estimate Std. Error t value Pr(>|t|)    
## (Intercept)     0.575460   0.008845  65.060  < 2e-16 ***
## post93         -0.002074   0.012931  -0.160  0.87261    
## anykids        -0.129498   0.011676 -11.091  < 2e-16 ***
## post93:anykids  0.046873   0.017158   2.732  0.00631 ** 
## ---
## Signif. codes:  0 '***' 0.001 '**' 0.01 '*' 0.05 '.' 0.1 ' ' 1
## 
## Residual standard error: 0.4967 on 13742 degrees of freedom
## Multiple R-squared:  0.0126, Adjusted R-squared:  0.01238 
## F-statistic: 58.45 on 3 and 13742 DF,  p-value: < 2.2e-16
\end{verbatim}

\begin{enumerate}
\def\labelenumi{\arabic{enumi}.}
\setcounter{enumi}{7}
\tightlist
\item
  Re-estimate this model including demographic characteristics as well
  as state and year fixed effect. Use the variable nonwhite, age, ed,
  and unearn as demographics.
\end{enumerate}

\begin{Shaded}
\begin{Highlighting}[]
\CommentTok{\#install.packages(\textquotesingle{}lfe\textquotesingle{})}
\FunctionTok{library}\NormalTok{(lfe)}
\end{Highlighting}
\end{Shaded}

\begin{verbatim}
## Loading required package: Matrix
\end{verbatim}

\begin{Shaded}
\begin{Highlighting}[]
\NormalTok{eitc}\SpecialCharTok{$}\NormalTok{interact }\OtherTok{\textless{}{-}}\NormalTok{ eitc}\SpecialCharTok{$}\NormalTok{post93}\SpecialCharTok{*}\NormalTok{eitc}\SpecialCharTok{$}\NormalTok{anykids}
\NormalTok{reg2 }\OtherTok{\textless{}{-}} \FunctionTok{felm}\NormalTok{(work }\SpecialCharTok{\textasciitilde{}}\NormalTok{ anykids}\SpecialCharTok{+}\NormalTok{interact}\SpecialCharTok{+}\NormalTok{nonwhite}\SpecialCharTok{+}\NormalTok{age}\SpecialCharTok{+}\NormalTok{ed}\SpecialCharTok{+}\NormalTok{unearn}\SpecialCharTok{|}\NormalTok{state}\SpecialCharTok{+}\NormalTok{year, }\AttributeTok{data=}\NormalTok{eitc)}
\FunctionTok{summary}\NormalTok{(reg2)}
\end{Highlighting}
\end{Shaded}

\begin{verbatim}
## 
## Call:
##    felm(formula = work ~ anykids + interact + nonwhite + age + ed +      unearn | state + year, data = eitc) 
## 
## Residuals:
##     Min      1Q  Median      3Q     Max 
## -0.9011 -0.4596  0.1867  0.4269  2.4472 
## 
## Coefficients:
##            Estimate Std. Error t value Pr(>|t|)    
## anykids  -0.1060901  0.0114251  -9.286  < 2e-16 ***
## interact  0.0526466  0.0163114   3.228  0.00125 ** 
## nonwhite -0.0806300  0.0095755  -8.420  < 2e-16 ***
## age       0.0033024  0.0004217   7.830 5.23e-15 ***
## ed        0.0155543  0.0015967   9.742  < 2e-16 ***
## unearn   -0.0174273  0.0005736 -30.382  < 2e-16 ***
## ---
## Signif. codes:  0 '***' 0.001 '**' 0.01 '*' 0.05 '.' 0.1 ' ' 1
## 
## Residual standard error: 0.4711 on 13684 degrees of freedom
## Multiple R-squared(full model): 0.1156   Adjusted R-squared: 0.1116 
## Multiple R-squared(proj model): 0.08644   Adjusted R-squared: 0.08237 
## F-statistic(full model):29.31 on 61 and 13684 DF, p-value: < 2.2e-16 
## F-statistic(proj model): 215.8 on 6 and 13684 DF, p-value: < 2.2e-16
\end{verbatim}

\begin{enumerate}
\def\labelenumi{\arabic{enumi}.}
\setcounter{enumi}{8}
\tightlist
\item
  Explain why you can't use finc, earn, and uearn in the same
  regression.
\end{enumerate}

Answer: You can't use finc, earn, and uearn variable in the same
regression, because you get perfect multicolinearity.

\begin{enumerate}
\def\labelenumi{\arabic{enumi}.}
\setcounter{enumi}{9}
\tightlist
\item
  Estimate a ``placebo'' treatment model. Take data from only the
  pre-reform period. Use the same treatment and control groups.
  Introduce a placebo policy that begins in 1992 instead of 1994 (so
  1992 and 1993 both have this fake policy).
\end{enumerate}

\begin{Shaded}
\begin{Highlighting}[]
\NormalTok{eitc}\SpecialCharTok{$}\NormalTok{post91 }\OtherTok{=} \FunctionTok{as.numeric}\NormalTok{(eitc}\SpecialCharTok{$}\NormalTok{year }\SpecialCharTok{\textgreater{}=} \DecValTok{1992}\NormalTok{)}

\NormalTok{reg3 }\OtherTok{\textless{}{-}} \FunctionTok{lm}\NormalTok{(work }\SpecialCharTok{\textasciitilde{}}\NormalTok{ post91}\SpecialCharTok{*}\NormalTok{anykids, }\AttributeTok{data =}\NormalTok{ eitc[eitc}\SpecialCharTok{$}\NormalTok{year }\SpecialCharTok{\textless{}}\DecValTok{1994}\NormalTok{, ])}

\NormalTok{cr\_reg3 }\OtherTok{\textless{}{-}} \FunctionTok{c}\NormalTok{(}\StringTok{"post93"} \OtherTok{=} \StringTok{"Post 93"}\NormalTok{, }\StringTok{"anykids"} \OtherTok{=} \StringTok{"Any Kids"}\NormalTok{,}\StringTok{"post93:anykids"}\OtherTok{=}\StringTok{"Post93: Any kids"}\NormalTok{, }\StringTok{"interact"}\OtherTok{=}\StringTok{"Interaction"}\NormalTok{, }\StringTok{"nonwhite"} \OtherTok{=} \StringTok{"Non{-}White"}\NormalTok{, }\StringTok{"age"} \OtherTok{=} \StringTok{"Age"}\NormalTok{, }\StringTok{"ed"} \OtherTok{=} \StringTok{"Years of Education"}\NormalTok{, }\StringTok{"unearn"} \OtherTok{=} \StringTok{"Unearned Income"}\NormalTok{,}\StringTok{"post91"} \OtherTok{=} \StringTok{"Post 91"}\NormalTok{,}\StringTok{"post91:anykids"}\OtherTok{=}\StringTok{"Post91: Any kids"}\NormalTok{)}
  
\FunctionTok{modelsummary}\NormalTok{(}\FunctionTok{list}\NormalTok{(}\StringTok{"Diff{-}in{-}diff"}\OtherTok{=}\NormalTok{reg1,}\StringTok{"Re{-}estimate"}\OtherTok{=}\NormalTok{reg2,}\StringTok{"Placebo"}\OtherTok{=}\NormalTok{reg3), }\AttributeTok{vcov =} \FunctionTok{c}\NormalTok{(}\StringTok{"robust"}\NormalTok{,}\StringTok{"iid"}\NormalTok{,}\StringTok{"robust"}\NormalTok{), }\AttributeTok{stars =} \ConstantTok{TRUE}\NormalTok{, }\AttributeTok{coef\_rename =}\NormalTok{ cr\_reg3)}
\end{Highlighting}
\end{Shaded}

\begin{verbatim}
## Warning: When the `vcov` argument is set to "iid", "classical", or "constant",
##   `modelsummary` extracts the default variance-covariance matrix from the
##   model object. For objects of class `felm`, the default vcov is not
##   always IID. Please make sure that the standard error label matches the
##   numeric results in the table. Note that the `vcov` argument accepts a
##   named list for users who want to customize the standard error labels in
##   their regression tables.
\end{verbatim}

\begin{table}
\centering
\begin{tabular}[t]{lccc}
\toprule
  & Diff-in-diff & Re-estimate & Placebo\\
\midrule
(Intercept) & \num{0.575}*** &  & \num{0.583}***\\
 & (\num{0.009}) &  & (\num{0.015})\\
Post 93 & \num{-0.002} &  & \\
 & (\num{0.013}) &  & \\
Any Kids & \num{-0.129}*** & \num{-0.106}*** & \num{-0.123}***\\
 & (\num{0.012}) & (\num{0.011}) & (\num{0.020})\\
Post93: Any kids & \num{0.047}** &  & \\
 & (\num{0.017}) &  & \\
Interaction &  & \num{0.053}** & \\
 &  & (\num{0.016}) & \\
Non-White &  & \num{-0.081}*** & \\
 &  & (\num{0.010}) & \\
Age &  & \num{0.003}*** & \\
 &  & (\num{0.000}) & \\
Years of Education &  & \num{0.016}*** & \\
 &  & (\num{0.002}) & \\
Unearned Income &  & \num{-0.017}*** & \\
 &  & (\num{0.001}) & \\
Post 91 &  &  & \num{-0.012}\\
 &  &  & (\num{0.018})\\
Post91: Any kids &  &  & \num{-0.010}\\
 &  &  & (\num{0.024})\\
\midrule
Num.Obs. & \num{13746} & \num{13746} & \num{7401}\\
R2 & \num{0.013} & \num{0.116} & \num{0.017}\\
R2 Adj. & \num{0.012} & \num{0.112} & \num{0.016}\\
AIC & \num{19779.8} & \num{18382.1} & \num{10628.4}\\
BIC & \num{19817.5} & \num{18856.4} & \num{10663.0}\\
Log.Lik. & \num{-9884.917} &  & \num{-5309.218}\\
RMSE & \num{0.50} & \num{0.47} & \num{0.50}\\
Std.Errors & HC3 & IID & HC3\\
\bottomrule
\multicolumn{4}{l}{\rule{0pt}{1em}+ p $<$ 0.1, * p $<$ 0.05, ** p $<$ 0.01, *** p $<$ 0.001}\\
\end{tabular}
\end{table}

\end{document}
